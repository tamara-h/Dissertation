
\section{Literature Review}
\subsection{Sentiment Analysis Tools}
\begin{itemize}
    \item Be critical, strengths and weaknesses of existing work
\end{itemize}

Many existing sentiment analysis tools are tailored towards market research for businesses, so focus on providing opinion mining tools for social media. This means that they focus on extracting the valence, and the subject that is being talked about in the text. An example of this is the MonkeyLearn sentiment analysis tool, whoic

There has been a little research into using a multi-dimensional VAD structure to investigate sentiment, primarily investigating whether using a VAD structure could be used to help identify burnout in software developers by analysing the messages in development repositories. \cite{mantyla2016mining}

\subsection{Sentiment Representation structures}





\subsection{Data Pre-Processing Approaches}
\subsection{Model building approaches}
\begin{itemize}
    \item How can textual sentiment prediction be optimised?
    \item Does using more than 1 dimension to classify emotions provide more insight, and how can this be quantified?
\end{itemize}