\documentclass{article}
\usepackage[utf8]{inputenc}
\usepackage{graphicx}
\usepackage{listings}
\usepackage{color}
\usepackage[margin=0.75in]{geometry}
\usepackage{multirow}
\usepackage{multicol}
\usepackage{todonotes}
\usepackage[toc,page]{appendix}
\usepackage[square,numbers]{natbib}
\usepackage[utf8]{inputenc} % Required for inputting international characters
\usepackage[T1]{fontenc} % Output font encoding for international characters

\graphicspath{}

\lstdefinestyle{leftCode}{
  belowcaptionskip=1\baselineskip,
  breaklines=true,
  frame=L,
  xleftmargin=\parindent,
  showstringspaces=false,
  basicstyle=\footnotesize\ttfamily
}

\newcommand\todoNote[1]{\textcolor{red}{#1}}

\begin{document}

%----------------------------------------------------------------------------------------
%	TITLE PAGE
%----------------------------------------------------------------------------------------

\begin{titlepage} % Suppresses displaying the page number on the title page and the subsequent page counts as page 1
	\newcommand{\HRule}{\rule{\linewidth}{0.5mm}} % Defines a new command for horizontal lines, change thickness here
	
	\center % Centre everything on the page
	
	%------------------------------------------------
	%	Headings
	%------------------------------------------------
	
	\textsc{\LARGE University of Birmingham}\\[1.5cm] % Main heading such as the name of your university/college
	
	\textsc{\Large Department of Computer Science}\\[0.5cm] % Major heading such as course name
	
	\textsc{\large Dissertation for BSc. Mathematics and Computer Science}\\[0.5cm] % Minor heading such as course title
	
	%------------------------------------------------
	%	Title
	%------------------------------------------------
	
	\HRule\\[0.4cm]
	
	{\huge\bfseries Semantic Analysis of Text}\\[0.4cm] % Title of your document
	
	\HRule\\[1.5cm]
	
	%------------------------------------------------
	%	Author(s)
	%------------------------------------------------

	% If you don't want a supervisor, uncomment the two lines below and comment the code above
	{\large\textit{Author}}\\
	Tamara \textsc{Herbert} \\ 
	1557437
	
	%------------------------------------------------
	%	Date
	%------------------------------------------------
	
	\vfill\vfill\vfill % Position the date 3/4 down the remaining page
	
	{\large April  2019} % Date, change the \today to a set date if you want to be precise
	
	%------------------------------------------------
	%	Logo
	%------------------------------------------------
	
	%\vfill\vfill
	\includegraphics[width=0.2\textwidth]{litImgs/birmingham.png}\\[1cm] % Include a department/university logo - this will require the graphicx package
	 
	%----------------------------------------------------------------------------------------
	
	\vfill % Push the date up 1/4 of the remaining page
	
\end{titlepage}

%-----------------

\tableofcontents

\pagebreak

\section{Introduction}

Analysing the emotions behind a piece of text is not always an easy problem even for human readers, and trying to compute this is much harder. Existing sentiment analysis tools primarily concentrate on just whether a text is positive or negative. \cite{kolchyna2015twitter}
Most text that we want to analyse however expresses more emotion than just positivity or negativity, and this project will attempt to use various analysis techniques to analyse input text and produce detailed sentimental feedback about it.

\subsection{Research Questions}
\begin{itemize}
    \item How can textual sentiment prediction be optimised?
    \item Does using more than 1 dimension to classify emotions provide more insight, and how can this be quantified?

\end{itemize}




\section{Literature Review}
\subsection{Sentiment Analysis Tools}
\begin{itemize}
    \item Be critical, strengths and weaknesses of existing work
\end{itemize}

Many existing sentiment analysis tools are tailored towards market research for businesses, so focus on providing opinion mining tools for social media. This means that they focus on extracting the valence, and the subject that is being talked about in the text. In terms of obtaining an emotion from a piece of text, this is just limited to retrieving how positive or negative it is. Emotion can be argued as more than just a binary sentiment, so there is an interest into whether more information about the text's mood can be analysed. 


\subsection{Sentiment Representation structures}

Almost all existing sentiment analysis tools give a resultant emotion back just with the valence.

To achieve a more in depth understanding of sentiment rather than just the valence, an argument can be made to add two more dimensions, giving the Valence-Arousal-Dominance (VAD) structure for representing emotion.



% \subsection{Data Pre-Processing Approaches}
\subsection{Model building approaches}
\begin{itemize}
    \item How can textual sentiment prediction be optimised?
    \item Does using more than 1 dimension to classify emotions provide more insight, and how can this be quantified?
\end{itemize}

There has been a little research into using a multi-dimensional VAD structure to investigate sentiment, primarily investigating whether using a VAD structure could be used to help identify burnout in software developers by analysing the messages in development repositories. \cite{mantyla2016mining} In this case, a correlation was found between each of the dimensions and different issues raised within the input text, meaning that there is an argument for using multiple dimensions to help understand textual data to a greater degree. An issue with this research is that they only used a word based lexicon where each word individual word was assigned a value. This loses the context in which each word is being used in, and by using a VAD dataset for sentences to train over this issue will be mitigated.



   
\section{Methodology}
\subsection{Data Imbalance}
\subsection{Hypothesis Tests}
\subsection{Data Preprocessing (N-Gram and Feature selection)}
\subsection{Model Selection}
\begin{itemize}
    \item Be critical, strengths and weaknesses of existing work
\end{itemize}
\subsection{Oversampling and Undersampling}
\subsection{Implementation}
\input{results.tex}
\section{Results and Discussion}
\begin{itemize}
    \item How can textual sentiment prediction be optimised?
    \item Does using more than 1 dimension to classify emotions provide more insight, and how can this be quantified?
\end{itemize}

\todo{mosaic plot, adjusting the score}

To be able to evoke enough emotive text from a user, the UI states "Please describe how your week is going so far. Be as detailed as possible.". This was arbitrarily chosen, and 

\subsection{Further Work}




\section{Conclusion}

\pagebreak


\begin{appendices}

\section{Hypothesis Test Results}
\label{appendix:hypothesis}
The R Script for running the relevant hypotheisis test over the data can be found at the following locations:
\begin{center}

% \begin{table}[h]
\begin{tabular}{|l|l|}
\hline
 Test &  Location\\ \hline
 Number of features &  sentimentAnalysis/statisticalAnalysis/featureSelection.R\\
 Model Selection &  sentimentAnalysis/statisticalAnalysis/modelSelection.R\\
 &  \\ \hline
\end{tabular}
% \end{table}
\end{center}

\section{Project File Structure}

\section{Project Log}

\end{appendices}



\bibliographystyle{unsrtnat}
\bibliography{ref,lit}


\end{document}
